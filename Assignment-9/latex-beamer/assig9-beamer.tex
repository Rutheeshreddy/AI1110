\documentclass{beamer}
\usepackage{listings}
\usepackage{blkarray}
\usepackage{listings}
\usepackage{subcaption}
\usepackage{url}
\usepackage{tikz}
\usepackage{tkz-euclide} % loads  TikZ and tkz-base
%\usetkzobj{all}
\usetikzlibrary{calc,math}
\usepackage{float}
\newcommand\norm[1]{\left\lVert#1\right\rVert}
\renewcommand{\vec}[1]{\mathbf{#1}}
\usepackage[export]{adjustbox}
\usepackage[utf8]{inputenc}
\usepackage{amsmath}
\usepackage{amsfonts}
\usepackage{tikz}
\usepackage{hyperref}
\usepackage{multirow}
\usepackage{bm}
\hypersetup{
    colorlinks = true,
    linkbordercolor = {white},
    linkcolor={red},
    citecolor={green},
    filecolor={blue},
	menucolor={red},
	runcolor={cyan},
	urlcolor={blue},
	breaklinks=true
}
\usetikzlibrary{automata, positioning}
\usetheme{Boadilla}
\providecommand{\pr}[1]{\ensuremath{\Pr\left(#1\right)}}
\providecommand{\mbf}{\mathbf}
\providecommand{\qfunc}[1]{\ensuremath{Q\left(#1\right)}}
\providecommand{\sbrak}[1]{\ensuremath{{}\left[#1\right]}}
\providecommand{\lsbrak}[1]{\ensuremath{{}\left[#1\right.}}
\providecommand{\rsbrak}[1]{\ensuremath{{}\left.#1\right]}}
\providecommand{\brak}[1]{\ensuremath{\left(#1\right)}}
\providecommand{\lbrak}[1]{\ensuremath{\left(#1\right.}}
\providecommand{\rbrak}[1]{\ensuremath{\left.#1\right)}}
\providecommand{\cbrak}[1]{\ensuremath{\left\{#1\right\}}}
\providecommand{\lcbrak}[1]{\ensuremath{\left\{#1\right.}}
\providecommand{\rcbrak}[1]{\ensuremath{\left.#1\right\}}}
\providecommand{\abs}[1]{\vert#1\vert}
\newcommand*{\permcomb}[4][0mu]{{{}^{#3}\mkern#1#2_{#4}}}
\newcommand*{\perm}[1][-3mu]{\permcomb[#1]{P}}
\newcommand*{\comb}[1][-1mu]{\permcomb[#1]{C}}

\newcounter{saveenumi}
\newcommand{\seti}{\setcounter{saveenumi}{\value{enumi}}}
\newcommand{\conti}{\setcounter{enumi}{\value{saveenumi}}}

\makeatletter
\newenvironment<>{proofs}[1][\proofname]{%
    \par
    \def\insertproofname{#1\@addpunct{.}}%
    \usebeamertemplate{proof begin}#2}
  {\usebeamertemplate{proof end}}
\makeatother
%% Theme choice:
%\usetheme{CambridgeUS}

% Title page details: 
\title{A Problem on Approximating Mean of a Function of a Random Variable} 
\author{Chittepu Rutheesh Reddy\\
 CS21BTECH11014}
\date{\today}
\logo{\large \LaTeX{}}


\begin{document}

% Title page frame
\begin{frame}
    \titlepage 
\end{frame}

% Remove logo from the next slides
\logo{}


% Outline frame
\begin{frame}{Outline}
    \tableofcontents
\end{frame}




% Lists frame
\section{Question}
\begin{frame}{Question}

\begin{block}
{\textbf{Q29 [12$^{th}$ Papoulis Textbook Exercise 5]:}}

Using the equation, \\
\begin{align}
 \eta_y = E(g(x)) \approx g(\eta) + g^{``}(\eta)\frac{\sigma^2}{2}
 \label{eq:1}
\end{align}
find $E(x^3)$ if $\eta_x = 10$ and $\sigma_x = 2$. 
\end{block}
\end{frame}


% Blocks frame
\section{Solution}

\begin{frame}{Solution}
 Before finding the answer, let us prove equation \eqref{eq:1} \\
\textbf{Proof}:
If $\textbf{y} = g(\textbf{x})$ is new random variable, formed from original random variable \textbf{x}, then 
\begin{align}
E(g(x)) = \int_{-\infty}^{\infty} g(x)f(x)dx \label{eq:2}
\end{align}
Generally, to determine this value, knowledge of f(x) is required. But if \textbf{x} is concentrated towards it's mean, then $E(g(\textbf{x}))$
can be approximated.\\
\par Suppose $f(x)$ is negligible outside the interval $(\eta - \epsilon,\eta + \epsilon)$ and $\epsilon$ is very small, such that in this interval $g(x)\approx g(\eta)$, then $g(x)$ can be approximated with
\begin{align}
 g(x) \approx g(\eta) + g^{`}(\eta)(x-\eta)+......g^{n}(\eta)\frac{(x-\eta)^n}{n!}
\end{align}
\end{frame} 

\begin{frame}{Solution}
 Approximating $g(x)$ to two degree polynomial by neglecting higher terms and putting into equation \eqref{eq:2} gives us
 \begin{align}
 \begin{split}
E(g(x)) &= g(\eta) \int_{-\infty}^{\infty}f(x)dx + g^{`}(\eta)\int_{-\infty}^{\infty}(x-\eta)f(x)dx\\
 &+ g^{``}(\eta)\int_{-\infty}^{\infty}\frac{(x-\eta)^2}{2}f(x)dx 
 \end{split} \\
E(g(x)) &= g(\eta) + g^{``}(\eta)\frac{\sigma^2}{2}
\end{align}
Given,
\begin{align}
g(x) &= x^3,\eta = 10,\sigma = 2 
 \end{align}
After substituting, $E(g(x)) = 1120$

\end{frame}


\end{document}