\documentclass{beamer}
\usepackage{listings}
\usepackage{blkarray}
\usepackage{listings}
\usepackage{subcaption}
\usepackage{url}
\usepackage{tikz}
\usepackage{tkz-euclide} % loads  TikZ and tkz-base
%\usetkzobj{all}
\usetikzlibrary{calc,math}
\usepackage{float}
\newcommand\norm[1]{\left\lVert#1\right\rVert}
\renewcommand{\vec}[1]{\mathbf{#1}}
\usepackage[export]{adjustbox}
\usepackage[utf8]{inputenc}
\usepackage{amsmath}
\usepackage{amsfonts}
\usepackage{tikz}
\usepackage{hyperref}
\usepackage{multirow}
\usepackage{bm}
\hypersetup{
    colorlinks = true,
    linkbordercolor = {white},
    linkcolor={red},
    citecolor={green},
    filecolor={blue},
	menucolor={red},
	runcolor={cyan},
	urlcolor={blue},
	breaklinks=true
}
\usetikzlibrary{automata, positioning}
\usetheme{Boadilla}
\providecommand{\pr}[1]{\ensuremath{\Pr\left(#1\right)}}
\providecommand{\mbf}{\mathbf}
\providecommand{\qfunc}[1]{\ensuremath{Q\left(#1\right)}}
\providecommand{\sbrak}[1]{\ensuremath{{}\left[#1\right]}}
\providecommand{\lsbrak}[1]{\ensuremath{{}\left[#1\right.}}
\providecommand{\rsbrak}[1]{\ensuremath{{}\left.#1\right]}}
\providecommand{\brak}[1]{\ensuremath{\left(#1\right)}}
\providecommand{\lbrak}[1]{\ensuremath{\left(#1\right.}}
\providecommand{\rbrak}[1]{\ensuremath{\left.#1\right)}}
\providecommand{\cbrak}[1]{\ensuremath{\left\{#1\right\}}}
\providecommand{\lcbrak}[1]{\ensuremath{\left\{#1\right.}}
\providecommand{\rcbrak}[1]{\ensuremath{\left.#1\right\}}}
\providecommand{\abs}[1]{\vert#1\vert}

\newcounter{saveenumi}
\newcommand{\seti}{\setcounter{saveenumi}{\value{enumi}}}
\newcommand{\conti}{\setcounter{enumi}{\value{saveenumi}}}

\makeatletter
\newenvironment<>{proofs}[1][\proofname]{%
    \par
    \def\insertproofname{#1\@addpunct{.}}%
    \usebeamertemplate{proof begin}#2}
  {\usebeamertemplate{proof end}}
\makeatother
%% Theme choice:
%\usetheme{CambridgeUS}

% Title page details: 
\title{A Problem On Total Probability Theorem} 
\author{Chittepu Rutheesh Reddy\\
 CS21BTECH11014}
\date{\today}
\logo{\large \LaTeX{}}


\begin{document}

% Title page frame
\begin{frame}
    \titlepage 
\end{frame}

% Remove logo from the next slides
\logo{}


% Outline frame
\begin{frame}{Outline}
    \tableofcontents
\end{frame}

% Lists frame
\section{Total Probability Theorem}
\begin{frame}{Total Probability Theorem}


\textbf{Statement:}

Let the events $E_1, E_2, E_3, ......E_n$ be a set of exhaustive events of a sample space S, such that {$E_1, E_2, E_3, ......E_n$} are partitions of a sample space S, the happening of a event A from the sample space S is\\

\begin{align}
 \pr{A} = \sum_{i=1}^n \pr{E_i}\pr{A|E_i}
\end{align}

\end{frame}


% Lists frame
\section{Question}
\begin{frame}{Question}

\begin{block}
{\textbf{Q1 [12$^{th}$ CBSE Probability Exercise 13.3]:}}

An urn contains 5 red and 5 black balls. A ball is drawn at random, its colour is noted and is returned to the urn. Moreover, 2 additional balls of the colour drawn are put in the urn and then a ball is drawn at random. What is the probability that the second ball is red?
\end{block}
\end{frame}


% Blocks frame
\section{Solution}
\begin{frame}{Solution}
Let $X \in \{0,1\}$ and $Y \in \{0,1\}$ be the random variables representing the outcomes defined as follows.\\
\begin{table}[ht!]
\resizebox{\columnwidth}{!}{
\begin{tabular}{|c|c|c|}
\hline
\textbf{Symbol} & \textbf{Formula} & \textbf{Definition} \\
\hline \hline
M(x) & $3x^2-10x+3$ &  Additional cost per unit when units are incremented\\
\hline
C(x) & \(\int M(x) \,dx\) & Total expenses in terms of units \\
\hline
A(x) & $\frac{C(x)}{x}$ & It is cost per unit \\
\hline
\end{tabular}
}

\caption{}
\label{table:table1}
\end{table}
\end{frame}
\begin{frame}{Solution}
 Given data of the question, in terms of probability is presented in the table 
 
 \begin{table}[ht!]
 \centering

%\resizebox{\columnwidth}{!}{
	\begin{tabular}{|c|c|}
		\hline
		\textbf{Probability}  &\textbf{Value} \\
                \hline
		\pr{Y=0|X=0} &$\frac{7}{12}$ \\
		\hline
		\pr{Y=0|X=1} &$\frac{5}{12}$ \\
		\hline
		\pr{X=0}  &$\frac{1}{2}$ \\
		\hline
		\pr{X=1}  &$\frac{1}{2}$ \\
		\hline
	\end{tabular}
	
%}
\caption{}
\label{table:table2}
\end{table}
\end{frame}
\begin{frame}{Solution}
The required probability is \pr{Y=0}.\\ 
By total probability theorem \\
\begin{align}
  \pr{Y=0} &= \sum_{i=0}^1 \pr{Y=0|X=i}\pr{X=i}\\
           &= \frac{7}{12}.\frac{1}{2} + \frac{5}{12}.\frac{1}{2}\\
           &= \frac{1}{2}    
\end{align}
  $\therefore$ The probability that second ball drawn is red is 0.5
\end{frame} 
\end{document}