\documentclass{beamer}
\usepackage{listings}
\usepackage{blkarray}
\usepackage{listings}
\usepackage{subcaption}
\usepackage{url}
\usepackage{tikz}
\usepackage{tkz-euclide} % loads  TikZ and tkz-base
%\usetkzobj{all}
\usetikzlibrary{calc,math}
\usepackage{float}
\newcommand\norm[1]{\left\lVert#1\right\rVert}
\renewcommand{\vec}[1]{\mathbf{#1}}
\usepackage[export]{adjustbox}
\usepackage[utf8]{inputenc}
\usepackage{amsmath}
\usepackage{amsfonts}
\usepackage{tikz}
\usepackage{hyperref}
\usepackage{multirow}
\usepackage{bm}
\hypersetup{
    colorlinks = true,
    linkbordercolor = {white},
    linkcolor={red},
    citecolor={green},
    filecolor={blue},
	menucolor={red},
	runcolor={cyan},
	urlcolor={blue},
	breaklinks=true
}
\usetikzlibrary{automata, positioning}
\usetheme{Boadilla}
\providecommand{\pr}[1]{\ensuremath{\Pr\left(#1\right)}}
\providecommand{\mbf}{\mathbf}
\providecommand{\qfunc}[1]{\ensuremath{Q\left(#1\right)}}
\providecommand{\sbrak}[1]{\ensuremath{{}\left[#1\right]}}
\providecommand{\lsbrak}[1]{\ensuremath{{}\left[#1\right.}}
\providecommand{\rsbrak}[1]{\ensuremath{{}\left.#1\right]}}
\providecommand{\brak}[1]{\ensuremath{\left(#1\right)}}
\providecommand{\lbrak}[1]{\ensuremath{\left(#1\right.}}
\providecommand{\rbrak}[1]{\ensuremath{\left.#1\right)}}
\providecommand{\cbrak}[1]{\ensuremath{\left\{#1\right\}}}
\providecommand{\lcbrak}[1]{\ensuremath{\left\{#1\right.}}
\providecommand{\rcbrak}[1]{\ensuremath{\left.#1\right\}}}
\providecommand{\abs}[1]{\vert#1\vert}
\newcommand*{\permcomb}[4][0mu]{{{}^{#3}\mkern#1#2_{#4}}}
\newcommand*{\perm}[1][-3mu]{\permcomb[#1]{P}}
\newcommand*{\comb}[1][-1mu]{\permcomb[#1]{C}}

\newcounter{saveenumi}
\newcommand{\seti}{\setcounter{saveenumi}{\value{enumi}}}
\newcommand{\conti}{\setcounter{enumi}{\value{saveenumi}}}

\makeatletter
\newenvironment<>{proofs}[1][\proofname]{%
    \par
    \def\insertproofname{#1\@addpunct{.}}%
    \usebeamertemplate{proof begin}#2}
  {\usebeamertemplate{proof end}}
\makeatother
%% Theme choice:
%\usetheme{CambridgeUS}

% Title page details: 
\title{Assignment} 
\author{Chittepu Rutheesh Reddy\\
 CS21BTECH11014}
\date{\today}
\logo{\large \LaTeX{}}


\begin{document}

% Title page frame
\begin{frame}
    \titlepage 
\end{frame}

% Remove logo from the next slides
\logo{}


% Outline frame
\begin{frame}{Outline}
    \tableofcontents
\end{frame}




% Lists frame
\section{Uniform Distribution}
\begin{frame}{Problem 1.3}

\begin{block}
{Question}
Find theoretical expression of $F_U(x)$
\end{block}
\end{frame}


\begin{frame}{Solution}
For uniform distribution , the pdf is a constant
Let,
\begin{align}
p_U(x) &= a, 0 \le x \le 1 \\
\text{then, } F_U(x) &= \int _0^x a dx\\
F_U(x) &= ax, 0 \le x \le 1 \\
\text{As, } F_U(1) &= 1 \implies a = 1 \\
\therefore F_U(x) &= x, 0 \le x \le 1
\end{align}
\end{frame} 

\section{Gaussian Distribution}

\begin{frame}{Problem 2.5}

\begin{block}
{Question}
\begin{align}
p_X(x) = \frac{1}{\sqrt{2\pi}}exp(\frac{-x^2}{2}), \infty < x < \infty
\end{align}
Calculate mean and variance of X.
\end{block}
\end{frame}

\begin{frame}{Solution}
Mean, $U(X) = \int _\infty^\infty xp_X(x).dx$ \\
\begin{align}
\text{So, } U(X) &= \int _{-\infty}^\infty x\frac{1}{\sqrt{2\pi}}e^{(\frac{-x^2}{2})}.dx \\
U(X) &= \frac{-1}{\sqrt{2\pi}}\int_{-\infty}^\infty e^t.dt, \text{where, }t=\frac{-x^2}{2} \\
U(X) = 0
\end{align}
As $E(X)$ = 0, $Var(X) = E(X^2)$ , So
\begin{align}
E(X^2) &= \int_{-\infty}^\infty x^2\frac{1}{\sqrt{2\pi}}e^{(\frac{-x^2}{2})}.dx \\
E(X^2) &= \frac{1}{\sqrt{2\pi}}\int_{-\infty}^\infty x^2e^{(\frac{-x^2}{2})}.dx  = \frac{1}{\sqrt{2\pi}}\frac{\sqrt{2\pi}}{1}\\
E(X^2) &= 1
\end{align}
\end{frame} 
\section{From Uniform to Other}

\begin{frame}{Problem 3.2}

\begin{block}
{Question}
Find the theoretical expression of $F_V(X)$, where
\begin{align}
V = -2ln(1 - U)
\end{align}
\end{block}
\end{frame}

\begin{frame}{Solution}
If Y = g(X), we know that $F_Y(y) = F_X(g^{-1}(y))$, here 
\begin{align}
X &= -2ln(1-Y) \\
ln(1-Y) &= e^{\frac{-X}{2}}\\
Y &= 1 - e^{\frac{-X}{2}} \\ 
F_V(X) &= F_U(1 - e^{\frac{-X}{2}}) 
\end{align}
 when , $0 \le 1 - e^{\frac{-X}{2}} \le 1$
 \begin{align}
 0 &\le e^{\frac{-X}{2}} \le 1 \\
  X &\ge 0 , \text{So,} \\ 
  F_V(X) &= 1 - e^{\frac{-X}{2}}, X \ge 0 \\
  F_V(X) &= 0 , X < 0 
 \end{align}
\end{frame} 
\end{document}